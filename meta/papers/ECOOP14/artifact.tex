% Template for one-page artifact description appendices in lncs volumes
% Written by Camil Demetrescu and Erik Ernst
% April 8, 2014

% ARTIFACT: do not change the following line
\clearpage\appendix\section{Artifact Description}

% ARTIFACT: please note that the rest of this file contains (1) LaTeX commands
% and environments that you must use in the way they are given here, and in the
% same order, each producing one section of text; (2) example data filling in
% those commands and environments with text to show how they may be used; you
% must preserve the former and replace the latter by information concerning
% your artifact; the comments describe each of these sections to indicate
% their purpose; also note that the entire appendix must fit on one page

% ARTIFACT: section on authors of the artifact; note that this may differ from
% the list of authors of the paper, e.g., you may wish to mention other people
% who contributed to the artifact, but who did not co-author the paper
\artifactauthors{Daniele Filaretti and Sergio Maffeis.}

% ARTIFACT: section giving a brief summary of the purpose of the artifact
\begin{summary}
  The provided package is designed to support repeatability of the experiments of the paper: 
  in particular, it allows users to test the \kphp{} interpreter and symbolic model checker on a variety of 
  examples, including the ones discussed in the paper. 
  We provide details on how to install and build \kphp{}, together with step-by-step instructions for running the examples and getting users started with their own experiments.
\end{summary}

% ARTIFACT: section on the contents of the artifact (code, data, etc.)
\begin{content}
  The artifact package includes:
  \begin{itemize}
  \item the complete source files;
  \item a build of the \K\ Framework (which is needed for running \kphp{});
  \item all the examples discussed in the paper;
  \item a self-contained Linux-based virtual machine which can be used to run the artifact, if a user does not want to install \kphp{} and \K\ directly;
  \item detailed instructions on how to build and use the artifact in {\tt index.html}.
  \end{itemize}
\end{content} 

% ARTIFACT: section containing links to sites (e.g., github) holding the
% latest version of the code, if any
\begin{getting}
  The artifact, endorsed by the Artifact Evaluation Committee of ECOOP'14,  is available free
  of charge as supplementary material of this paper on SpringerLink. \kphp{} is still under development. The latest version of \kphp{}, together with a user-friendly
  web interface, is available online at \url{http://www.phpsemantics.org}.
\end{getting} 

% ARTIFACT: section specifying the platforms on which the artifact is known to
% work, including requirements beyond the operating system such as large
% amounts of memory or many processor cores
\begin{platforms}
  The \K\ tool binaries (needed for running \kphp{})
  run best on Linux and OS X. On Windows, Cygwin emulation must be used, 
  which may slow down execution. 
  The self-contained Linux-based VMware virtual machine we provide
  (included in the package) can be used in case of installation problems
  on non-standard system configurations. 
\end{platforms}

% ARTIFACT: section specifying the license under which the artifact is
% made available
\license{EPL-1.0 (\url{http://www.eclipse.org/legal/epl-v10.html}).}

% ARTIFACT: section specifying the md5 sum of the artifact master file
% uploaded to SpringerLink, enabling downloaders to check that the file is the
% expected version and suffered no damage during download
\mdsum{ed21cb3680e42078bc7cd2778ecfe03f.}

% ARTIFACT: section specifying the size of the artifact master file uploaded
% to SpringerLink
\artifactsize{281.6 MB.}

% ARTIFACT: include here any additional references, if needed...

%\bibliographystyle{unsrt}
%\begin{thebibliography}{10}
%...
%\end{thebibliography}

